\documentclass[a4paper, 11pt]{article}

% Compiler avec XeLaTeX !!

\usepackage{hyperref}

\usepackage{amsmath}
\usepackage{amsthm}
\usepackage{amssymb}
\usepackage{stmaryrd}
\usepackage[shortlabels]{enumitem}

\usepackage{unicode-math}
\usepackage{polyglossia}
\setmainlanguage{french}
\usepackage{csquotes} % guillemets

\usepackage{tikz}
\usetikzlibrary{shapes,arrows}

% amsthm config

\newtheorem{théorème}{Théorème}
\newtheorem{proposition}[théorème]{Proposition}
\newtheorem{lemme}[théorème]{Lemme}
\newtheorem{corollaire}[théorème]{Corollaire}
\theoremstyle{definition}
\newtheorem{définition}{Définition}

% Notations
% \mathbb instead of \mathbf because
% http://www.madore.org/~david/weblog/d.2009-02-08.1608.html

\newcommand{\nat}{\mathbb{N}}
\newcommand{\integer}{\mathbb{N}}
\newcommand{\real}{\mathbb{R}}
\newcommand{\complex}{\mathbb{C}}
\newcommand{\aff}{\mathbb{A}}
\newcommand{\proj}{\mathbb{P}}

\newcommand{\carac}{\textrm{char}}

\newcommand{\derivp}[2]{\dfrac{\partial #1}{\partial #2}}
\newcommand{\derivpp}[2]{\dfrac{\partial^2 #1}{\partial {#2}^2}}
\newcommand{\derivpmix}[3]{\dfrac{\partial^2 #1}{\partial #2 \partial #3}}

\begin{document}

\title{Introduction à la géométrie algébrique}
\author{Cours donné par Ilia Itenberg}
\date{2015--2016}

\maketitle

Page web du cours : \url{http://mathfond.math.upmc.fr/2015-16/fiches/Itenberg-intro.html}

Le but de ce cours est de fournir beaucoup d'exemples, pour préparer
aux cours plus avancés de géométrie algébrique.

Référence :
\begin{itemize}
\item \textit{Basic Algebraic Geometry}, I. Shafarevich (le début du cours est calqué là-dessus)
\item Pour l'algèbre commutative : \textit{Commutative Algebra}, O. Zariski et P. Samuel
\end{itemize}

Progression du cours :
\begin{enumerate}
\item Lundi 07/09 : Courbes algébriques, courbes rationnelles, corps
  de fonctions rationnelles.
\item Vendredi 11/09 : Paramétrages. Applications rationnelles,
  équivalence birationnelle.
\item Lundi 14/09 : Polygones de Newton. Équations de Weierstrass de
  courbes (hyper)elliptiques. Points et courbes singuliers.
\item Vendredi 18/09 : Paramètres locaux, multiplicité
  d'intersection. Projectivisation des courbes algébriques.
\item Lundi 21/09 : Cartes affines de $\proj^2$ (??). Applications
  rationnelles dans le cadre projectif. Énoncé du théorème de Bézout.
\item Vendredi 25/09 : Preuve du théorème de Bézout et application au
  théorème de Pascal. Variétés algébriques et topologie de Zariski.
\item Lundi 28/09 : Nullstellensatz. Espaces topologiques nœthériens,
  composantes irréductibles, dimension de Krull.
\end{enumerate}

\tableofcontents

\newpage

\section{Courbes affines planes}

\subsection{Définition et premiers exemples}

Voici une première tentative de définir la notion de \enquote{courbe
  définie par une équation polynomiale} :
\begin{définition}
  Une \emph{courbe algébrique plane} définie sur un corps $K$ est
  l'ensemble des points $(x,y) \in K^2$ tels que $f(x,y) = 0$, où $f$
  est un polynôme non constant à deux variables et à coefficients dans
  $K$.
\end{définition}
Note : le plan $K^2$ sera souvent noté $\aff^2$, c'est-à-dire l'espace
affine de dimension 2 (sous-entendu : sur le corps K)

Soit $C = \{(x,y)\in K^2 \mid f(x,y)=0\}$. On peut définir :
\begin{itemize}
\item le degré de $C$ : c'est le degré du polynôme $f$
\item la notion d'irréductibilité : $C$ est une courbe irréductible si
  $f$ est un polynôme irréductible.
\end{itemize}
Précisions l'interprétation géométrique de l'irréductibilité : si
$f \in K[x,y]$ est non constant, on peut le décomposer\footnote{De
  façon unique, à permutation et à multiplication par un scalaire
  près.} en facteurs irréductibles :
\[ f = f_1^{k_1} \times \ldots \times f_r^{k_r} \]
Dans ce cas, en posant $C_i = {f_i = 0}$, on a :
\[ C = C_1 \cup \ldots \cup C_r \]
autrement dit, la courbe se décompose en courbes plus simples. Une
courbe irréductible, moralement, ne peut pas se décomposer ainsi.

\paragraph{Cependant,} la définition 1 n'est pas toujours satisfaisante !
Ainsi, considérons $x^2 + y^2 = 0$ sur $\real^2$ ($K = \real$).
\begin{itemize}
\item La courbe est réduite au point $(0,0)$, ce qui intuitivement,
  est un objet de dimension 0 et non 1 !
\item Quel est le degré de cette courbe ? Le même ensemble étant
  défini par $x^4 + y^4 = 0$, il n'y a pas unicité du polynôme dont
  l'ensemble est lieu d'annulation ; le degré est donc mal défini (de
  même pour l'irréductibilité).
\end{itemize}

Pour résoudre ces difficultés, on va vouloir prendre $K$les notions
\emph{algébriquement clos}. Afin d'en constater les avantages,
prouvons d'abord le lemme suivant :
\begin{lemme}
  Soit $K$ un corps\footnote{À ce stade, on ne demande pas au corps
    d'être algébriquement clos. Il peut même être fini, auquel cas la
    conclusion est trivialement vraie.}, $f \in K[x,y]$ irréductible,
  $g \in K[x,y]$ quelconque.

  Alors, si $f \nmid g$, alors le système de deux équations
  $f(x,y) = 0, g(x,y) = 0$ n'a qu'un nombre fini de solutions.
\end{lemme}
\begin{proof}
  \emph{Astuce} : considérer $f$ et $g$ comme polynômes à une variable
  à coefficients dans\footnote{Rappel : c'est le corps des fractions
    rationnelles en une indéterminée $y$.} $K(y)$, i.e.
  $f, g \in K(y)[x]$. On remarque que
  \begin{enumerate}[(a)]
  \item $f$ est irréductible dans $K(y)[x]$
  \item $f \nmid g$ dans $K(y)[x]$
  \end{enumerate}
  Esquisse de preuve du $(i)$ (le $(ii)$ étant semblable) : supposons
  que $f = pq$ dans $K[x,y]$. En multipliant par le dénominateur
  commun des coefficients de $p$ dans $K(y)$, puis par celui de
  $q$, on obtient
  \[ a(y)f(x,y) = p'(x,y)q'(x,y) \]
  où $p', q' \in K[x,y]$ et $a \in K[y]$. Ce qui signifie que
  $f \mid p'q'$ dans $K[x,y]$. Or, par hypothèse, $f$ est irréductible
  dans $K[x,y]$, qui est un anneau factoriel. Donc, parmi $p$ et $q$,
  l'un des deux est divisible par $f$ et l'autre divise donc $a(y)$,
  c'est-à-dire qu'il est dans $K[y]$ : dans $K(y)[x]$, il a le statut
  de constante. On a montré que toute factorisation de $f$ dans
  $K(y)[x]$ était triviale.

  Une fois $(i)$ et $(ii)$ établis, on sait que $f$ et $g$ sont
  premiers entre eux dans $K(y)[x]$, donc il existe $u, v \in K(y)[x]$
  tels que $uf+vg=1$. Encore une fois, on multiplie par le
  dénominateur commun et on aboutit à une égalité de la forme
  \[ u'(x,y)f(x,y) + v'(x,y)g(x,y) = h(y) \]
  où $u', v' \in K[x,y]$ et $h \in K[y]$.

  Si $(x_0, y_0)$ est solution du système, alors $y_0$ est racine du
  polynôme à une variable $h$. Il existe donc un nombre fini de $y_0$
  possibles, et pour chacun d'entre eux, ???
\end{proof}

On se souvient maintenant que les corps algébriquement clos sont tous
\emph{infinis} (exercice), ce qui permet d'obtenir :
\begin{corollaire}
  Sur un corps $K$ algébriquement clos, tout polynôme
  \emph{irréductible} dans $K[x,y]$ est déterminé par sa courbe dans
  $\aff^2$ de façon unique, à un facteur constant près.
\end{corollaire}
\begin{proof}
  Toute courbe $\{f(x,y)=0\}$ dans $\aff^2$ sur un corps
  algébriquement clos a un nombre infini de points : si on fixe $x_0$
  de sorte que $f(x_0,y)$ ne soit pas constant non nul en $y$, il
  existe forcément une racine $y_0$, et il y a une infinité de $x_0$
  possibles ? (on utilise donc deux fois le fait que $K$ soit
  algébriquement clos).

  Ainsi, soit $f \in K[x,y]$ irréductible, et $g \in K[x,y]$
  définissant la même courbe. Ils ont alors une infinité de zéros
  communs, donc d'après le lemme précédent, $f \mid g$.
\end{proof}

Ceci permet un choix canonique de polynôme pour chaque courbe
irréductible. On généralise aux courbes quelconques par un résultat
analogue sur les polynômes \emph{sans facteur multiple}. On a donc, en
restreignant les corps considérés, résolu les soucis avec la
définition (une alternative aurait été de définir algébriquement la
notion de courbe de façon plus sophistiquée).

\paragraph{Un autre avantage des corps algébriquement clos} : le nombre de
racines\footnote{Comptées avec multiplicité, bien sûr.} d'un polynôme
est \emph{exactement} égal au degré du polynôme. Plus généralement,
sans clôture algébrique, on perd beaucoup d'information sur le nombre
d'intersections entre courbes. En témoigne cet énoncé approximatif du
\emph{théorème de Bezout}, qu'on reverra plus tard :

\textit{Soient $C_1$ et $C_2$ deux courbes dans $\aff^2$, de degrés
  respectifs $d_1$ et $d_2$. Si $C_1 \cap C_2$ est fini, alors son
  cardinal est majoré par $d_1 d_2$, et c'est même une égalité sous
  certaines conditions de généricité.}

Dans $\real^2$, ceci est \emph{faux}, on peut penser par exemple aux
nombreux cas pour l'intersection de deux coniques. Attention,
\emph{même la majoration est fausse} ! Elle devient vraie si on
remplace l'hypothèse donnée, c'est-à-dire \enquote{le nombre de points
  d'intersection sur $\real$ est fini}, par \enquote{il n'existe pas
  de \emph{composante commune}} (ceci sera vu plus loin), le premier
n'impliquant pas le second. (\textit{Exercice} : trouver un contre-exemple.)

\paragraph{}À partir de maintenant, \textbf{sauf mention du contraire, le corps de
  base sera pris algébriquement clos}. Ce cadre permet quand même de
faire de la géométrie sur $\real$, en travaillant dans $\complex$ tout
en préservant l'invariance par la conjugaison complexe.

\paragraph{Un exemple \enquote{scolaire}} : la droite polaire d'un point et d'un
cercle.

(Faire un dessin.) La construction géométrique par les
tangentes échoue lorsque $P$ est à l'intérieur du
cercle. Algébriquement, on trouve deux points de tangence dans
$\complex\setminus\real$, qui sont \emph{conjugués}. La droite
complexe qui les relie dans $\complex^2$ intersecte $\real^2$ et on
retrouve une droite polaire dans le plan usuel, \emph{algébriquement}.

\paragraph{}Un autre cas intéressant qui dépasse le cadre algébriquement clos (et
ne sera donc pas traité dans ce cours) est $K = \mathbb{Q}$. Exemple
célèbre : la courbe de Fermat $x^n + y^n = 1$, qui n'admet pas de
points rationnels pour $n \geq 3$.

\subsection{Courbes rationnelles}

\emph{Attention}, la terminologie ne signifie \emph{pas} que ces
courbes sont définies sur $\mathbb{Q}$ !

\paragraph{Exemple} Considérons la courbe définie par
$y^2 = x^2 + x^3$. Il s'agit d'une cubique irréductible, qui possède
en plus la propriété intéressante suivante : on peut en donner un
paramétrage par deux fonctions rationnelles d'une variable.

Géométriquement, ce qu'on va faire, c'est paramétrer la courbe par la
pente de la droite reliant le point considéré à l'origine ; on peut
voir ça aussi comme une projection sur la droite $x=1$, de centre
l'origine.

FAIRE UN DESSIN

Concrètement, on va poser $y = tx$. On a alors $t^2 x^2 = x^2 +
x^3$. Il y a deux cas à distinguer :
\begin{itemize}
\item $x \neq 0$ : on simplifie, d'où $t^2 = 1 + x$, puis $x = t^2 -
1$ et $y = t(t^2 - 1)$.
\item $x=0$ : OK, $(0,0)$ est un point de la courbe, et on le retrouve
avec le paramétrage précédent lorsque $t=0$.
\end{itemize}
Finalement on obtient bien quelque chose du type
$(x,y) = (\phi(t), \psi(t))$ où $\phi, \psi \in K(t)$.

\begin{définition}
  Soit $C$ une courbe irréductible dans $\aff^2$, définie par un
  polynôme irréductible $f$. On dit que $C$ est \emph{rationnelle}
  s'il existe deux fonctions rationnelles $\phi$ et $\psi$, dont au
  moins une non constante, telles que $f(\phi(t),\psi(t)) = 0$ pour
  tout $t \in K$ où $\phi$ et $\psi$ sont toutes deux définies.
  
  $(\phi, \psi)$ est appelée une \emph{paramétrisation}\footnote{C'est
    comme ça que c'est dit dans le cours, mais \enquote{paramétrage}
    serait peut-être préférable ?} de $C$.
\end{définition}

\paragraph{Exemples} (1) Toute droite dans $\aff^2$ est rationnelle (exercice).\\
(2) Toute conique irréductible dans $\aff^2$ est rationnelle.

Démontrons le (2). (FAIRE UN DESSIN avec une ellipse qui sera
suggestif mais aura peu de sens car on ne travaille en aucun cas sur
$\real$ !)

Soit $P_0 = (x_0, y_0) \in C$, une droite passant par ce point coupe $C$ en
un autre point (sauf pour une droite particulière, la tangente). On va
paramétrer par la pente de la droite, en posant $y-y_0 =
t(x-x_0)$. Substituer $y$ par $y_0 + t(x-x_0)$ dans $f(x,y)$ donne
\[ A(t)x^2 + B(t)x + C(t) = 0, \quad A,B,C \in K[t] \]
Pour $t$ fixé, la droite de pente $t$ passant par $P_0$ intersecte la
conique en deux points, $P_0$ d'abscisse $x_0$, et le point $P$ qu'on
veut paramétrer d'abscisse $x$ : ce sont les deux racines du trinôme
ci-dessus. Par relation coefficients-racines, on en déduit $x$ comme
une fonction rationnelle de $t$, puis $y$ en découle.

Dans le cas particulier de la conique, on voit que le paramétrage est
très intéressant : connaître \emph{un} point rationnel permet de
connaître \emph{tous} les points rationnels.

\paragraph{Quelles courbes sont rationnelles ?} Reformulation (????)
avec le corps des fonctions rationnelles.

Soit $C$ une courbe irréductible dans $\aff^2$ (lieu d'annulation du
polynôme irréductible $f$), on va définir un corps $K(C)$ contenant
les \enquote{fonctions rationnelles} définies sur $C$. On considère
les fonctions rationnelles
\[ u(x,y) = \frac{p(x,y)}{q(x,y)} \text{ telles que } f \nmid q \]
Notre but étant d'évaluer $u$ en des points de $C$, on souhaite que le
dénominateur ne soit pas identiquement nul sur $C$. Posons
\[ u_1 = \frac{p_1}{q_1} \sim u_2 = \frac{p_2}{q_2} \text{ ssi } f
\mid (p_1q_2 - p_2q_1 \]
On vérifie qu'il s'agit d'une relation d'équivalence, et que le
quotient de notre sous-ensemble de $K(x,y)$ par $\sim$ est bien un
corps. De plus, si $u_1 \sim u_2$, pour tout $p \in C$ où $u_1$ et
$u_2$ sont toutes définies, on aura $u_1(p) = u_2(p)$.
\begin{définition}
  Ce quotient est appelé \emph{corps des fonctions rationnelles sur
    $C$} et noté $K(C)$.
\end{définition}
\emph{Attention}, on peut avoir $u_1 \sim u_2$, $u_1$ définie en $p$,
et $u_2$ \emph{non} définie en $p$ !

\paragraph{Exemple} Sur le cercle unité $x^2 + y^2 = 1$, on peut
vérifier que
\[ u_1(x,y) = \frac{1-y}{x} \sim \frac{x}{1+y} = u_2(x,y) \]
Au point $(0,1)$, $u_1$ n'est pas définie, mais $u_2$ l'est, je peux
donc évaluer la classe de $u_1$ et $u_2$ dans $K(C)$ au point $(0,1)$
en choisissant le représentant $u_2$.

\begin{définition}
  Soit $C$ une courbe irréductible sur un corps $K$. On dit qu'un point
  $p \in C$ est \emph{régulier} pour un élément $\xi \in K(C)$ si,
  dans la classe d'équivalence $\xi$, on peut choisir un représentant
  $g$ qui soit défini en $p$.
\end{définition}

\paragraph{Exercice}
Considérons la courbe $C$ définie par $y = 0$.
Montrer que $K(C)$ peut être identifié à $K(x)$ (corps des fonctions
rationnelles en une indéterminée $x$).

Cette courbe nous fournit aussi un exemple de point non régulier :
$\frac{1}{x}$ n'est pas définie en $0$, et aucun autre représentant de
sa classe n'est défini en $0$.

\begin{proposition}
  Soit $C \subset \aff^2(K)$ une courbe irréductible. Alors $C$ est
  rationnelle si et seulement si $K(C)$ est isomorphe\footnote{Il
    s'agit d'un isomorphisme d'extensions de $K$, c'est-à-dire d'un
    isomorphisme de corps qui fixe le sous-corps commun $K$.} à
  $K(t)$.
\end{proposition}
\textit{Remarque} (pour ceux qui connaissent les extensions de
corps)~: $K(C)$ est de degré de transcendance 1 sur $K$. La
proposition affirme que $C$ est rationnelle ssi l'extension est
transcendante pure.

\begin{proof}
  \textit{Sens $\Rightarrow$.} On considère une paramétrisation
  $\phi(t), \psi(t)$ de notre courbe $C$. On va construire un
  plongement $\iota : K(C) \hookrightarrow K(t)$. Ce qu'on a envie de
  faire, c'est poser
  \[ \iota : K(C) \ni u(x,y) = \frac{p(x,y)}{q(x,y)} \mapsto
  \frac{p(\phi(t),\psi(t))}{q(\phi(t),\psi(t))} \]
  Il faut s'assurer que cette application est bien définie, donc
  \begin{itemize}
  \item qu'elle est compatible avec la relation d'équivalence : c'est
    immédiat en écrivant $f(\phi(t),\psi(t)) = 0$
  \item que $q(\phi(t),\psi(t))$ n'est pas identiquement nulle : il
    suffit pour cela de remarquer qu'il existe un nombre infini de
    points de $C$ de la forme $(\phi(t),\psi(t))$, en effet
    \begin{itemize}
    \item l'intersection des domaines de définition de $\phi$ et
      $\psi$ est infinie (même cofinie, car intersection de deux
      parties cofinies : les dénominateurs de $\phi$ et $\psi$ n'ont
      qu'un nombre fini de racines)
    \item chaque point de $C$ a un nombre fini d'antédédents par
      $(\phi, \psi)$
    \end{itemize}
  \end{itemize}
  Le morphisme défini fixe clairement $K$. En particulier, il n'est
  pas identiquement nul, donc il est injectif (son domaine étant un
  corps). \emph{Attention}, il n'est pas forcément \emph{surjectif} !
  Cependant, il réalise un isomorphisme entre $K(C)$ et un sous-corps
  $\iota(K(C)) \subset K(t)$ ; pour conclure, il suffit d'appliquer le
  \emph{théorème de Lüroth}\footnote{Il ne sera pas démontré en cours,
    on se référera à Zariski--Samuel pour une preuve.} : toute
  sous-extension $K \subsetneq L \subset K(t)$ est isomorphe à $K(t)$.

  \textit{Sens $\Leftarrow$.} On suppose que $K(C) \simeq K(t)$. Les
  (classes d'équivalences des) fonctions rationnelles $x$ et $y$ dans
  $K(C)$ sont envoyées par l'isomorphisme respectivement sur
  $\phi(t), \psi(t) \in K(t)$, au moins l'une des deux n'étant pas
  constante.

  Soit $f$ le polynôme irréductible définissant $C$. Dans $K(C)$, on a
  $f(x,y)=0$, cette égalité étant à lire comme \enquote{le polynôme
    $f$ évalué en les éléments $x, y \in K(C)$ a pour valeur la
    fonction rationnelle nulle}. En transférant par l'isomorphisme on
  a $f(\phi(t),\psi(t))=0$ dans $K(t)$. Ce qui montre que
  $(\phi(t), \psi(t))$ est une paramétrisation de $C$.
\end{proof}

Ce dernier sens de la démonstration donne un paramétrage particulier
de $C$, qui vérifie deux propriétés importantes :
\begin{enumerate}[(1)]
\item Tout point de $C$, sauf un nombre fini, peut être représenté
  comme $(\phi(t),\psi(t))$ pour une certaine valeur de $t \in K$.
\item À l'exception d'un nombre fini de points, une telle
  représentation est unique.
\end{enumerate}
Pour prouver cela, posons $\chi(x,y)$ qui correspond à $t$ par
l'isomorphisme. Alors
\begin{enumerate}[(i)]
\item $(x,y) = (\phi(\chi(x,y)), \psi(\chi(x,y)))$
\item $t = \chi(\phi(t),\psi(t))$
\end{enumerate}
L'égalité $(i)$ prouve $(1)$ car $\chi$ n'a qu'un nombre fini de
points de non-définition sur $C$, en vertu du lemme 1 (considérer le
système $f(x,y) = \chi(x,y) = 0$). De la même façon, $(2)$ découle de
$(ii)$.

\paragraph{Remarque} Ces propriétés ne sont absolument pas vérifiées
par \emph{toutes} les paramétrisations ! Mais on peut, à partir d'une
paramétrisation quelconque, récupérer une \enquote{bonne}
paramétrisation en effectuant la démonstration dans le sens
$(\Rightarrow)$, en utilisant l'isomorphisme du théorème de Lüroth, et
en appliquant le sens $(\Leftarrow)$. Si on vérifie déjà $(1)$ et
$(2)$, $\iota$ est surjectif (en fait, c'est équivalent), et on
retombe sur nos pas (exercice, où on pourra utiliser le fait que les
sous-extensions de $K(t)/K$ sont de la forme $K(g(t))$, c'est un
corollaire du théorème de Lüroth).

\paragraph{Exemple} On reprend l'exemple $y^2 = x^2 + x^3$, qu'on
avait paramétré par $x = t^2 - 1$, $y = t(t^2 - 1)$. Cette
paramétrisation vérifie les propriétés $(1)$ et $(2)$, mais on peut la
rendre plus mauvaise : par exemple, faire $t \leftarrow t^2$ (en
obtenant ainsi $x = t^4 - 1$, etc.) nous fait perdre l'unicité de la
représentation. Rien ne nous interdit, dans la définition d'un
paramétrage, de parcourir la courbe deux fois !

\subsection{Applications rationnelles}

\begin{définition}
  Soient $C$ et $D$ deux courbes irréductibles dans
  $\aff^2(K)$. Soient $u, v \in K(C)$. Si pour tout $p \in C$ régulier
  pour $u$ et $v$, $(u(p), v(p)) \in D$, on dit que $\Phi = (u,v)$ est
  une \emph{application rationnelle} de $C$ dans $D$.
\end{définition}
\paragraph{Remarque/notation} Une application rationnelle n'est pas
forcément une application (totale) au sens ensembliste ! On utilisera
la notation $\Phi : C \dashrightarrow D$ pour garder à l'esprit que
$\Phi$ peut ne pas être définie sur certains points.

\paragraph{Exemples} à traiter en exercice :
\begin{itemize}
\item La projection d'une conique irréductible sur une droite, à
  partir d'un point appartenant à la conique, est une application
  rationnelle.
\item Toute paramétrisation d'une courbe algébrique est une
  application rationnelle $\aff^1 \dashrightarrow C$.
\end{itemize}

\begin{définition}
  Soient $C$ et $D$ deux courbes irréductibles dans $\aff^2$. On dit
  qu'une application rationnelle $\Phi : C \dashrightarrow D$ est un
  \emph{isomorphisme birationnel} s'il existe une application
  rationnelle $\Psi : D \dashrightarrow C$ telle que $\Phi \circ \Psi$
  et $\Psi \circ \Phi$ coïncident avec $id_D$ et $id_C$ respectivement
  partout où elles sont définies, avec $\Psi$ non constante.
\end{définition}
\paragraph{Remarque} Cette dernière condition \enquote{$\Psi$ non
  constante} est là pour exclure des cas dégénérés où on ne souhaite
pas dire que $C$ et $D$ sont birationnellement isomorphes. Avec cette
définition, on montre alors que si $\Phi$ est un isomorphisme
birationnel, $\Phi$ est non constant (c'est bien ça ?) Autre
remarque~: si $\Psi : D \dashrightarrow C$ est constante, elle est
définie partout, car on peut choisir un représentant de la classe
d'équivalence qui est défini partout. (Tout ceci est du pinaillage pas
très intéressant.)

\begin{proposition}
  Si $\Phi : C \dashrightarrow D$ est un isomorphisme birationnel,
  alors pour tout point $q \in D$, $\Phi^{-1}(q)$ est fini.
\end{proposition}
\begin{proof}
  Exercice.
\end{proof}
\begin{corollaire}
  Si $\Phi : C \dashrightarrow D$ est un isomorphisme birationnel,
  alors $\Phi$ établit une bijection entre
  \[ U \cap \Phi^{-1}(V) \text{ et } V \cap \Psi^{-1}(U) \]
  et leurs complémentaires respectifs dans $C$ et $D$ sont finis, où
  \begin{itemize}
  \item $\Psi$ est la réciproque
  \item $U \subset C$ est le domaine de définition de $\Phi$
  \item $V \subset D$ est le domaine de définition de $\Psi$
  \end{itemize}
\end{corollaire}

\begin{définition}
  S'il existe un isomorphisme birationnel entre deux courbes
  irréductibles $C$ et $D$, on dit que $C$ et $D$ sont
  \emph{birationnellement équivalentes}.
\end{définition}

\paragraph{Exemple} à traiter en exercice :

Soit $C$ une courbe irréductible dans $\aff^2$. Alors $C$ est
rationnelle si et seulement si $C$ est birationnellement équivalente à
une droite. (Toutes les droites étant birationnellement équivalentes
entre elles, on pourrait remplacer \enquote{une droite} par
\enquote{la droite définie par $y=0$} dans cet énoncé.)

\textit{Indication} : Si $C$ est rationnelle, on peut considérer une
paramétrisation particulière de $C$ fournie par un isomorphisme entre
$K(C)$ et $K(t)$. Écrire alors $t = \chi(x,y)$ et considérer
$(x,y) \mapsto (\chi(x,y), 0)$.

Dans la suite, nous allons nous intéresser à des polynômes sur
lesquels nous restreindrons les monômes pouvant apparaître. Nous
utiliserons alors la représentation graphique suivante :
\begin{définition}
  Le \emph{polygone de Newton} d'un polynôme $f \in K[x,y]$ est
  l'enveloppe convexe dans $\real^2$ de tous les points
  $(i,j) \in \nat^2$ tels que $f$ contienne un monôme $a_{ij} x^i y^j$
  ($a_{ij} \neq 0$).
\end{définition}
Bien entendu, le polygone de Newton de $f$ est inclus dans le triangle
de sommets $(0,0)$, $(0,d)$ et $(d,0)$ où $d$ est le degré de $f$.

\paragraph{Exercice} Considérons une courbe irréductible de degré $d$
dans $\aff^2$, définie par $f(x,y)=0$. On suppose que $f$ ne contienne
que des monômes de degré $d-1$ et $d$. (FAIRE UN DESSIN du polygone de
Newton). Alors $C$ est rationnelle.

Remarquons qu'en particulier, l'exemple précédent $y^2 = x^2 + x^3$
satisfait ces conditions ; reprendre la même technique pour démontrer
le cas général.

\paragraph{Note} : dans le cas où tous ses monômes ont pour degré $d$
(le polygone de Newton est alors un segment), $f$ se décompose en
produit de facteurs linéaires (rappel : $K$ est toujours supposé
algébriquement clos). Or $f$ est irréductible, donc il est de degré 1.

Maintenant, on considère un polynôme irréductible $f \in K[x,y]$ dont
chaque monôme est de degré $d$, $d-1$ ou $d-2$. (DESSIN) Effectuons le
changement de coordonnées $y = tx$, autrement dit, appliquons
l'isomorphisme birationnel $(x,y) \mapsto (x, t = y/x)$ ; puis
divisons par $x^{d-2}$. On obtient alors
\[ A(t)x^2 + B(t)x + C(t) = 0, \qquad A,B,C \in K[t] \]
La courbe définie par cette équation dans le plan de coordonnées
$(x,t)$ est birationnellement équivalente à la courbe de départ.

Ce nouveau polynôme présente l'avantage d'avoir une variable qui n'est
que de degré 2, et on sait résoudre les équations du second
degré. Avec toutefois quelques précautions : la fonction racine carrée
n'est pas rationnelle ! En posant
\[ s = 2A(t)x + B(t) \quad\text{ et }\quad p(t) = B(t)^2 -
4A(t)C(t) \]
on réalise un isomorphisme birationnel (à condition que \emph{la
  caractéristique de $K$ soit différente de 2}) avec la courbe définie
(dans un plan de coordonnées $(s,t)$) par
\[ s^2 = p(t), \qquad p \in K[t] \]
Une telle courbe est appelée \emph{courbe hyperelliptique}.

\paragraph{Remarque} Supposons que $p$ soit de degré pair :
$\deg p = 2n,\; n \in \nat^*$. On peut alors faire baisser de 1 le degré
de l'équation. Pour cela, posons $p(t) = q(t)(t-\alpha)$ ($\alpha$
racine de $p$). Alors
\[ \left(\frac{s}{(t-\alpha)^n}\right)^n =
\frac{q(t)}{(t-\alpha)^{2n-1}} \]
ce qui se réécrit comme $(s')^2 = h(t')$ où $t' = \frac{1}{t-\alpha}$ et
$\deg h = \deg q = 2n-1$.

\paragraph{Cas des cubiques ($d=3$)} Les isomorphismes birationnels
successifs nous ramènent à $s^2 = p(t)$ où $\deg p \leq 4$. La
remarque précédente permet de faire baisser le degré à 3, on a
alors\footnote{On se remet ici à utiliser les noms $x$ et $y$ pour les
  coordonnées, ce qui ne signifie pas qu'on soit revenus à la courbe
  de départ !}
\[ y^2 = ax^3 + bx^2 + cx + d \]
équation appelée \emph{forme de Weierstrass} de la cubique
irréductible. Si $\carac K \neq 3$, on peut encore réécrire
$y^2 = x^3 + px + q$.

Dans le cas d'une cubique, la condition au départ sur les degrés
impose seulement l'absence d'un terme constant dans le polynôme,
c'est-à-dire que le point $(0,0)$ soit sur la courbe. Ce qu'on peut
toujours avoir quitte à effectuer une translation (qui est clairement
un isomorphisme birationnel). Tout ceci fonctionne donc pour n'importe
quelle cubique irréductible.

\begin{proposition}
  Soient $C$ et $D$ deux courbes irréductibles dans $\aff^2$. Alors
  elles sont birationnellement équivalentes si et seulement si
  $K(C) \simeq K(D)$.
\end{proposition}
\begin{proof}
  Laissée en exercice.
\end{proof}

\subsection{Points lisses et points singuliers}

\begin{définition}
  Soit $C$ une courbe dans $\aff^2$ définie par $f(x,y)=0$, et
  $P \in C$ un point. On dit que $P$ est un \emph{point singulier} de $C$ si
  \[ \derivp{f}{x}(P) = 0 \quad\text{ et }\quad
  \derivp{f}{y}(P) = 0 \]

  Sinon, on dit que $P$ est un point \emph{lisse} ou\footnote{Deux
    terminologies différentes désignant exactement la même chose.}
  \emph{non singulier} de $C$.
\end{définition}
\begin{définition}
  Si tous les points de $C$ sont non singuliers, on dit que $C$ est
  une courbe \emph{non singulière}.
\end{définition}

\paragraph{Interprétation pour $(0,0)$} On suppose que $(0,0)$ est un
point de $C$, i.e. $f$ n'a pas de terme constant. C'est un point
singulier si et seulement s'il n'existe pas de monôme de degré 1 dans
$f$, c'est-à-dire que son polygone de Newton est inclus dans
conv((0,d),(0,2),(2,0),(d,0)) (DESSIN).

\begin{proposition}
  Soit $C \subset \aff^2$ une courbe irréductible. Alors $C$ n'a qu'un
  nombre fini de points singuliers.
\end{proposition}
\begin{proof}
  Soit $f(x,y)=0$ une équation de $C$. Un point $P \in C$ est singulier ssi
  \[ f(P) = \derivp{f}{x}(P) = \derivp{f}{y}(P) = 0 \]
  Supposons maintenant qu'il existe un nombre infini de points
  singuliers. D'après le lemme 1, $f$ divise ses deux dérivées
  partielles, qui sont pourtant de degré inférieur. Donc
  \[ \derivp{f}{x}(P) = \derivp{f}{y} = 0 \]
  Si $\carac K = 0$, on obtient que $f$ est une constante. Sinon,
  $\carac K = p > 0$ et dans ce cas, tous les monômes, étant annulés
  par la dérivation, sont de la forme $a_{i,j}x^{pi}y^{pj}$ :
  \[ f(x, y) = \sum_{i,j} a_{i,j}x^{pi}y^{pj} = \left(\sum_{i,j}
    a_{i,j}^{1/p}x^{i}y^{j}\right)^p \]
  ($K$ étant supposé algébriquement clos, on choisit une racine
  $p$-ième pour chaque $a_{i,j}$, et on utilise l'endomorphisme de
  Frobenius). Ceci contredit l'irréductibilité de $f$.
\end{proof}

Si le polygone de Newton de $f$ est dans [trapèze avec (0,r) et (r,0),
DESSIN] on dit que $P = (0,0)$ est un point singulier \emph{d'ordre
  $\geq r$}.

\begin{définition}
  L'\emph{ordre} $r$ d'un polynôme $f$ / d'un point $P$ sur une courbe
  $\{f=0\}$ dans le cas où $P = (0,0)$ est le plus grand $r$ tel que
  $f$ ne contienne pas de monôme de degré $< r$.
\end{définition}
\begin{définition}
  Un point d'ordre exactement 2 est appelé \emph{point double}.
\end{définition}

Soit $\{f=0\}$ une courbe admettant $(0,0)$ comme point double. La
partie homogène de degré 2 de $f$ se décompose en deux facteurs
linéaires : $(\lambda_1 x + \mu_1 y) (\lambda_2 x + \mu_2 y)$. On a
alors deux cas :
\begin{itemize}
\item Les deux facteurs sont les mêmes, c'est-à-dire qu'on peut écrire
  le produit $a(\lambda x + \mu y)^2$ : dans ce cas, le point double
  est dit \emph{dégénéré}. 
\item Les deux facteurs sont distincts, il s'agit d'un point double
  \emph{non dégénéré}.
\end{itemize}

\paragraph{Exemples} avec dessins des graphes dans $\real^2$ à faire.
\begin{enumerate}
\item $y^2 = x^2 + x^3$ (dessiner polygone de Newton).\\
  $(0,0)$ est un point double non dégénéré : la partie de degré 2 est
  $y^2 - x^2 = (y-x)(y+x)$, les deux facteurs sont différents si
  $\carac K \neq 2$.
\item $y^2 = x^3$. Le polygone de Newton est réduit à un segment
  (DESSIN).\\
  $(0,0)$ est un point double dégénéré car la partie de
  degré 2 s'écrit $y \times y$. Dans $\real^2$, on observe un
  point de rebroussement\footnote{Ou \textit{cusp} en anglais.}
\end{enumerate}

\paragraph{Exercices}
\begin{enumerate}
\item Montrer que toute conique \emph{irréductible} est non
  singulière.
\item Considérons une cubique dans $\aff^2$ définie par une équation
  de Weierstrass $y^2 = ax^3 + bx^2 + cx + d = p(x)$. Montrer qu'elle
  est non singulière si et seulement si $p$ est un polynôme à racines
  simples.
\end{enumerate}

(Fin du cours 3, début du cours 4)

\begin{théorème}
  Soit $C \subset \aff^2$ une courbe irréductible, $P \in C$ un point
  \emph{lisse}. Alors il existe une fonction rationnelle $t$ sur $C$
  vérifiant :
  \begin{itemize}
  \item $t$ est régulière en $P$
  \item $t(P) = 0$
  \item pour tout $u \in K(C)\setminus\{0\}$, il existe
    $k \in \integer$ et $v \in K(C)$ régulière en $P$, $v(P)\neq 0$,
    tel que $u = t^k v$.
  \end{itemize}
  De plus, $u$ est régulière en $P$ si et seulement si $k \geq 0$.
\end{théorème}

\begin{définition}
  Une telle fonction rationnelle $t$ s'appelle \emph{paramètre local}
  en $P$.
\end{définition}

\paragraph{Remarque} Si $t$ et $t'$ sont deux paramètres locaux en $P$
alors $t' = tw$ où $w \in K(C)$ est régulière en $P$ et $w(P) \neq
0$. Par conséquent, le nombre $k$ apparaissant dans l'énoncé ne dépend
pas du choix d'un paramètre local.

\begin{définition}
  Ce nombre $k$ s'appelle la \emph{multiplicité de zéro} de $u$ en $P$.
\end{définition}
\begin{proof}
  Sans perte de généralité, prenons $P = (0,0)$ et plaçons-nous dans
  le cas $\derivp{f}{x}(P) \neq 0$. On va montrer que la fonction $x$
  convient comme choix de $t$.

  À COMPLÉTER.
\end{proof}

\paragraph{Multiplicité d'intersection} Soient $C, D \subset \aff^2$
deux courbes définies respectivement par $f(x,y) = 0$ et $g(x,y) =
0$.
On suppose que $f$ est irréductible et que $f \not\mid g$. $g$ définit
alors une fonction rationnelle non identiquement nulle sur $C$.

\begin{définition}
  Soit $P \in C \cap D$ un point lisse sur $C$. La \emph{multiplicité
    d'intersection} de $C$ et\footnote{La définition n'étant pas
    symétrique en $C$ et $D$, on a affaire à un \enquote{et}
    syntaxique qui n'est pas commutatif \texttt{:p}} $D$ en $P$, notée
  $m_P(C, D)$, est la multiplicité de zéro de $g \in K(C)$ en $P$.
\end{définition}

\paragraph{Exemple} Soit $C$ une droite dans $\aff^2$,
$P = (\alpha, \beta) \in C$. On peut alors paramétrer la droite comme
\[
\begin{cases}
  x = \alpha + t\xi \\
  y = \beta + t\eta
\end{cases}
\]
Soit $D$ une courbe définie par $g(x,y)=0$ passant par
$P$. \\ \textit{Exercice} : montrer que
\begin{enumerate}
\item $t$ est un paramètre local de $C$ en $P$
\item si $P$ est un point singulier de $D$ alors $m_P(C,D) \geq 2$
\item si $P$ est un point lisse de $D$ alors 
  \begin{enumerate}[(a)]
  \item $m_P(C,D) \geq 2$ si et seulement si $C$ est la \emph{droite
      tangente} de $D$ en $P$ (voir définition ci-dessous, TODO
    ajouter références dans le document).
  \item $m_P(C,D) \geq 2$ (on dit alors que $P$ est un \emph{point de
      flexion} de $D$) si et seulement si
\[
\begin{vmatrix}
\derivpp{g}{x} & \derivpmix{g}{x}{y} & \derivp{g}{x} \\
\derivpmix{g}{y}{x} & \derivpp{g}{y} & \derivp{g}{y} \\
\derivp{g}{x}       & \derivp{g}{y}  & 0
\end{vmatrix}
(P) = 0
\]
  \end{enumerate}
\end{enumerate}

\begin{définition}
  La droite tangente à une courbe $D$ définie par $g(x,y)=0$ en un
  point lisse $P = (\alpha, \beta)$ est la courbe définie par
  \[ \derivp{g}{x}(P)(x-\alpha) + \derivp{g}{y}(P)(y-\beta) = 0 \]
\end{définition}

\subsection{Courbes algébriques projectives}

\paragraph{Motivation} Il s'agit de se placer dans le bon cadre pour
pouvoir énoncer le théorème de Bézout\footnote{C'est la même
  motivation qui nous a poussé à définir précisément les multiplictés
  d'intersections à la section précédente.}, dont on peut constater
l'échec dans $\aff^2$ avec le cas de deux droites parallèles : le
produit des degrés est 1 et il n'y a pas d'intersection ! Dans le plan
projectif, deux droites parallèles se croisent en un point à l'infini.

Exemples :
\begin{itemize}
\item $\proj^2(\real) = S^2/\{\pm 1\}$
\item $\proj^2(\complex)$ est une variété topologique de dimension
  $4$. \textit{Fait} : $\proj^2(\complex)$ quotienté par la
  conjugaison complexe est difféomorphe à $S^4$ 
\end{itemize}

blah blah blah à mettre dans un document commun avec le cours de
surfaces de Riemann.

\begin{théorème}[de Bézout]
  Soient $C$ et $D$ telles que $C$ soit irréductible et lisse, et
  $C \not\subset D$. Alors la somme des multiplicités d'intersection
  de $C$ et $D$ est égale au produit de leurs degrés.
\end{théorème}
\begin{proof}
  Nous allons démontrer le théorème dans deux cas particuliers :
  \begin{enumerate}
  \item $\deg C = 1$
  \item $\deg C = 2$
  \end{enumerate}

\textit{Cas 1.} $C$ est une droite. 

\textit{Cas 2.} $C$ est une conique lisse.
\end{proof}

\begin{corollaire}[Théorème de Pascal]
  Soit $C$ une conique lisse dans $P^2$, on considère un hexagone
  inscrit. Les 3 points d'intersections des paires de (droites
  prolongeant des) côtés opposés sont alors alignés.
\end{corollaire}
DESSINS.
\begin{proof}
  Comme sur le dessin, on note $(L_i, M_i)$ pour $i=1,2,3$ les paires
  de côtés opposés. La droite $L_i$ (resp. $M_i$) est définie par
  $l_i = 0$ (resp. $m_i = 0$). On pose alors le
  pinceau\footnote{Famille de courbes à un paramètre, linéaire en ce
    paramètre.} de cubiques suivant, de paramètre $\lambda$ :
  \[ D_{\lambda} : l_1 l_2 l_3 + \lambda m_1 m_2 m_3 = 0 \]
  Soit $P \in C\setminus\{\textrm{hexagone}\}$. Il existe
  $\lambda$ tel que $P \in D_{\lambda}$. En effet, l'équation
  ci-dessus, à $P$ fixé, est linéaire en $\lambda$ et à coefficient
  directeur non nul puisque $P \not\in M_i$ pour $i=1,2,3$ ; elle
  admet donc une (unique) solution.

  Par l'absurde, si $C \not\subset D_\lambda$, alors $C$ et
  $D_\lambda$ satisfont les hypothèses du théorème de Bézout et ont
  donc 6 points d'intersections (comptés avec multiplicité). Or
  $D_\lambda$ passe par les sommets de l'hexagone ainsi que $P$, ce
  qui fait déjà 7 points distincts dans $C \cap D_\lambda$,
  contradiction.

  Puisque $C \subset D_\lambda$ et que $\deg D_\lambda - \deg C = 1$,
  $D_\lambda$ est une courbe réductible et contient en fait deux
  composantes : une conique, $C$, et une droite qu'on notera $L$.
  Autrement dit $D_\lambda = C \cup L$ or les $L_i \cap M_i$ sont dans
  $D_\lambda \setminus C$ donc ils sont alignés et $L$ est la droite
  qui les relie.
\end{proof}

\section{Variétés algébriques affines}

\subsection{Définitions}

\begin{définition}
  \emph{lieu des zéros}
\end{définition}

\begin{définition}
  \emph{sous-ensemble algébrique fermé}
\end{définition}

\begin{proposition}
  stabilité par blabla
\end{proposition}
\begin{proof}
  Évident.
\end{proof}

Autrement dit on a une topologie sur $\aff^n$ dans laquelle les
ensembles fermés sont exactement des sous-ensembles algébriques
fermés.

\begin{définition}
  Cette topologie s'appelle la \emph{topologie de Zariski} de $\aff^n$.
\end{définition}

\subsection{Théorème des zéros de Hilbert}

\begin{théorème}[des zéros de Hilbert, ou Nullstellensatz]
  Soit $\mathcal{a} \subset K[x_1,\dots,x_n]$ un idéal et soit
  $f \in K[x_1,\dots,x_n]$ tels que $Z(\mathcal{a}) \subset
  Z(f)$. Alors $f \in \sqrt{\mathcal{a}}$.
\end{théorème}

\paragraph{Cas particulier} Nullstellensatz faible.

\begin{proof}
  Astuce de Rabinowitsch.
\end{proof}

\end{document}

