\documentclass[a4paper, 11pt]{article}

% Compiler avec XeLaTeX !!

\usepackage{hyperref}

\usepackage{amsmath}
\usepackage{amsthm}
\usepackage{amssymb}
\usepackage{stmaryrd}
\usepackage[shortlabels]{enumitem}

\usepackage{unicode-math}
\usepackage{polyglossia}
\setmainlanguage{french}
\usepackage{csquotes} % guillemets

\usepackage{tikz}
\usetikzlibrary{shapes,arrows}

% amsthm config

\newtheorem{théorème}{Théorème}
\newtheorem{propriété}[théorème]{Propriété}
\newtheorem{lemme}[théorème]{Lemme}
\newtheorem{corollaire}[théorème]{Corollaire}
\theoremstyle{definition}
\newtheorem{définition}{Définition}

% Notations
% \mathbb instead of \mathbf because
% http://www.madore.org/~david/weblog/d.2009-02-08.1608.html

\newcommand{\nat}{\mathbb{N}}
\newcommand{\real}{\mathbb{R}}
\newcommand{\complex}{\mathbb{C}}
\newcommand{\aff}{\mathbb{A}}
\newcommand{\proj}{\mathbb{A}}

\begin{document}

\title{Introduction à la géométrie algébrique}
\author{Cours donné par Ilia Itenberg}
\date{2015--2016}

\maketitle

Page web du cours : \url{http://mathfond.math.upmc.fr/2015-16/fiches/Itenberg-intro.html}

Le but de ce cours est de fournir beaucoup d'exemples, pour préparer
aux cours plus avancés de géométrie algébrique

Référence : \textit{Basic Algebraic Geometry} de Shafarevich ?

\begin{itemize}
\item Cours 1 : sections blabla à blabla
\item Cours 2 : sections blabla à blabla
\end{itemize}

\tableofcontents

\newpage

\section{Courbes affines planes}

\subsection{Définition et premiers exemples}

Voici une première tentative de définir la notion de \enquote{courbe
  définie par une équation polynomiale} :
\begin{définition}
  Une \emph{courbe algébrique plane} définie sur un corps $K$ est
  l'ensemble des points $(x,y) = K$ tels que $f(x,y) = 0$, où $f$ est
  un polynôme non constant à deux variables et à coefficients dans
  $K$.
\end{définition}
Note : le plan $K^2$ sera souvent noté $\aff^2$, c'est-à-dire l'espace
affine de dimension 2 (sous-entendu : sur le corps K)

Soit $C = \{(x,y)\in K^2 \mid f(x,y)=0\}$. On peut définir :
\begin{itemize}
\item le degré de $C$ : c'est le degré du polynôme $f$
\item la notion d'irréductibilité : $C$ est une courbe irréductible si
  $f$ est un polynôme irréductible.
\end{itemize}
Précisions l'interprétation géométrique de l'irréductibilité : si
$f \in K[x,y]$ est non constant, on peut le décomposer\footnote{De
  façon unique, à permutation et à multiplication par un scalaire
  près.} en facteurs irréductibles :
\[ f = f_1^{k_1} \times \ldots \times f_r^{k_r} \]
Dans ce cas, en posant $C_i = {f_i = 0}$, on a :
\[ C = C_1 \cup \ldots \cup C_r \]
autrement dit, la courbe se décompose en courbes plus simples. Une
courbe irréductible, moralement, ne peut pas se décomposer ainsi.

Cependant, la définition 1 n'est pas toujours satisfaisante !
Ainsi, considérons $x^2 + y^2 = 0$ sur $\real^2$ ($K = \real$).
\begin{itemize}
\item La courbe est réduite au point $(0,0)$, ce qui intuitivement,
  est un objet de dimension 0 et non 1 !
\item Quel est le degré de cette courbe ? Le même ensemble étant
  défini par $x^4 + y^4 = 0$, il n'y a pas unicité du polynôme dont
  l'ensemble est lieu d'annulation ; le degré est donc mal défini (de
  même pour l'irréductibilité).
\end{itemize}

Pour résoudre ces difficultés, on va vouloir prendre $K$les notions
\emph{algébriquement clos}. Afin d'en constater les avantages,
prouvons d'abord le lemme suivant :
\begin{lemme}
  Soit $K$ un corps\footnote{À ce stade, on ne demande pas au corps
    d'être algébriquement clos. Il peut même être fini, auquel cas la
    conclusion est trivialement vraie.}, $f \in K[x,y]$ irréductible,
  $g \in K[x,y]$ quelconque.

  Alors, si $f \nmid g$, alors le système de deux équations
  $f(x,y) = 0, g(x,y) = 0$ n'a qu'un nombre fini de solutions.
\end{lemme}
\begin{proof}
  \emph{Astuce} : considérer $f$ et $g$ comme polynômes à une variable
  à coefficients dans\footnote{Rappel : c'est le corps des fractions
    rationnelles en une indéterminée $y$.} $K(y)$, i.e.
  $f, g \in K(y)[x]$. On remarque que
  \begin{enumerate}[(a)]
  \item $f$ est irréductible dans $K(y)[x]$
  \item $f \nmid g$ dans $K(y)[x]$
  \end{enumerate}
  Esquisse de preuve du $(i)$ (le $(ii)$ étant semblable) : supposons
  que $f = pq$ dans $K[x,y]$. En multipliant par le dénominateur
  commun des coefficients de $p$ dans $K(y)$, puis par celui de
  $q$, on obtient
  \[ a(y)f(x,y) = p'(x,y)q'(x,y) \]
  où $p', q' \in K[x,y]$ et $a \in K[y]$. Ce qui signifie que
  $f \mid p'q'$ dans $K[x,y]$. Or, par hypothèse, $f$ est irréductible
  dans $K[x,y]$, qui est un anneau factoriel. Donc, parmi $p$ et $q$,
  l'un des deux est divisible par $f$ et l'autre divise donc $a(y)$,
  c'est-à-dire qu'il est dans $K[y]$ : dans $K(y)[x]$, il a le statut
  de constante. On a montré que toute factorisation de $f$ dans
  $K(y)[x]$ était triviale.

  Une fois $(i)$ et $(ii)$ établis, on sait que $f$ et $g$ sont
  premiers entre eux dans $K(y)[x]$, donc il existe $u, v \in K(y)[x]$
  tels que $uf+vg=1$. Encore une fois, on multiplie par le
  dénominateur commun et on aboutit à une égalité de la forme
  \[ u'(x,y)f(x,y) + v'(x,y)g(x,y) = h(y) \]
  où $u', v' \in K[x,y]$ et $h \in K[y]$.

  Si $(x_0, y_0)$ est solution du système, alors $y_0$ est racine du
  polynôme à une variable $h$. Il existe donc un nombre fini de $y_0$
  possibles, et pour chacun d'entre eux, ???
\end{proof}

On se souvient maintenant que les corps algébriquement clos sont tous
\emph{infinis} (exercice), ce qui permet d'obtenir :
\begin{corollaire}
  Sur un corps $K$ algébriquement clos, tout polynôme
  \emph{irréductible} dans $K[x,y]$ est déterminé par sa courbe dans
  $\aff^2$ de façon unique, à un facteur constant près.
\end{corollaire}
\begin{proof}
  Toute courbe $\{f(x,y)=0\}$ dans $\aff^2$ sur un corps
  algébriquement clos a un nombre infini de points : si on fixe $x_0$
  de sorte que $f(x_0,y)$ ne soit pas constant non nul en $y$, il
  existe forcément une racine $y_0$, et il y a une infinité de $x_0$
  possibles ? (on utilise donc deux fois le fait que $K$ soit
  algébriquement clos).

  Ainsi, soit $f \in K[x,y]$ irréductible, et $g \in K[x,y]$
  définissant la même courbe. Ils ont alors une infinité de zéros
  communs, donc d'après le lemme précédent, $f \mid g$.
\end{proof}

Ceci permet un choix canonique de polynôme pour chaque courbe
irréductible. On généralise aux courbes quelconques par un résultat
analogue sur les polynômes \emph{sans facteur multiple}. On a donc, en
restreignant les corps considérés, résolu les soucis avec la
définition (une alternative aurait été de définir algébriquement la
notion de courbe de façon plus sophistiquée).\\

Un autre avantage des corps algébriquement clos : le nombre de
racines\footnote{Comptées avec multiplicité, bien sûr.} d'un polynôme
est \emph{exactement} égal au degré du polynôme. Plus généralement,
sans clôture algébrique, on perd beaucoup d'information sur le nombre
d'intersections entre courbes. En témoigne cet énoncé approximatif du
\emph{théorème de Bezout}, qu'on reverra plus tard :

\textit{Soient $C_1$ et $C_2$ deux courbes dans $\aff^2$, de degrés
  respectifs $d_1$ et $d_2$. Si $C_1 \cap C_2$ est fini, alors son
  cardinal est majoré par $d_1 d_2$, et c'est même une égalité sous
  certaines conditions de généricité.}

Dans $\real^2$, ceci est \emph{faux}, on peut penser par exemple aux
nombreux cas pour l'intersection de deux coniques. Attention,
\emph{même la majoration est fausse} ! Elle devient vraie si on
remplace l'hypothèse donnée, c'est-à-dire \enquote{le nombre de points
  d'intersection sur $\real$ est fini}, par \enquote{il n'existe pas
  de \emph{composante commune}} (ceci sera vu plus loin), le premier
n'impliquant pas le second. (\textit{Exercice} : trouver un contre-exemple.)\\

À partir de maintenant, \textbf{sauf mention du contraire, le corps de
  base sera pris algébriquement clos}. Ce cadre permet quand même de
faire de la géométrie sur $\real$, en travaillant dans $\complex$ tout
en préservant l'invariance par la conjugaison complexe.

Un exemple \enquote{scolaire} : la droite polaire d'un point et d'un
cercle. (Faire un dessin.) La construction géométrique par les
tangentes échoue lorsque $P$ est à l'intérieur du
cercle. Algébriquement, on trouve deux points de tangence dans
$\complex\setminus\real$, qui sont \emph{conjugués}. La droite
complexe qui les relie dans $\complex^2$ intersecte $\real^2$ et on
retrouve une droite polaire dans le plan usuel, \emph{algébriquement}.

Un autre cas intéressant qui dépasse le cadre algébriquement clos (et
ne sera donc pas traité dans ce cours) est $K = \mathbb{Q}$. Exemple
célèbre : la courbe de Fermat $x^n + y^n = 1$, qui n'admet pas de
points rationnels pour $n \geq 3$.

\subsection{Courbes rationnelles}

\emph{Attention}, la terminologie ne signifie \emph{pas} que ces
courbes sont définies sur $\mathbb{Q}$ !

Considérons comme exemple la courbe définie par $y^2 = x^2 + x^3$. Il
s'agit d'une cubique irréductible, qui possède en plus la propriété
intéressante suivante : on peut en donner un paramétrage par deux
fonctions rationnelles d'une variable.

Géométriquement, ce qu'on va faire, c'est paramétrer la courbe par la
pente de la droite reliant le point considéré à l'origine ; on peut
voir ça aussi comme une projection sur la droite $x=1$, de centre
l'origine.

FAIRE UN DESSIN

Concrètement, on va poser $y = tx$. On a alors $t^2 x^2 = x^2 +
x^3$. Il y a deux cas à distinguer :
\begin{itemize}
\item $x \neq 0$ : on simplifie, d'où $t^2 = 1 + x$, puis
  $x = t^2 - 1$ et $y = t(t^2 - 1)$.
\item $x=0$ : OK, $(0,0)$ est un point de la courbe, et on le retrouve
  avec le paramétrage précédent lorsque $t=0$.
\end{itemize}
Finalement on obtient bien quelque chose du type
$(x,y) = (\phi(t), \psi(t))$ où $\phi, \psi \in K(t)$.

\begin{définition}
Une courbe rationnelle  
\end{définition}

\end{document}

