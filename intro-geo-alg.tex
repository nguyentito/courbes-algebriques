\documentclass[a4paper, 11pt]{article}

% Compiler avec XeLaTeX !!

\usepackage{hyperref}

\usepackage{amsmath}
\usepackage{amsthm}
\usepackage{amssymb}
\usepackage{stmaryrd}
\usepackage[shortlabels]{enumitem}

\usepackage{unicode-math}
\usepackage{polyglossia}
\setmainlanguage{french}
\usepackage{csquotes} % guillemets

\usepackage{tikz}
\usetikzlibrary{shapes,arrows}

% amsthm config

\newtheorem{théorème}{Théorème}
\newtheorem{propriété}[théorème]{Propriété}
\newtheorem{lemme}[théorème]{Lemme}
\newtheorem{corollaire}[théorème]{Corollaire}
\theoremstyle{definition}
\newtheorem{définition}{Définition}

% Notations
% \mathbb instead of \mathbf because
% http://www.madore.org/~david/weblog/d.2009-02-08.1608.html

\newcommand{\nat}{\mathbb{N}}
\newcommand{\real}{\mathbb{R}}
\newcommand{\complex}{\mathbb{C}}
\newcommand{\aff}{\mathbb{A}}
\newcommand{\proj}{\mathbb{P}}

\begin{document}

\title{Introduction à la géométrie algébrique}
\author{Cours donné par Ilia Itenberg}
\date{2015--2016}

\maketitle

Page web du cours : \url{http://mathfond.math.upmc.fr/2015-16/fiches/Itenberg-intro.html}

Le but de ce cours est de fournir beaucoup d'exemples, pour préparer
aux cours plus avancés de géométrie algébrique

Référence :
\begin{itemize}
\item \textit{Basic Algebraic Geometry} de Shafarevich ?
\item Pour l'algèbre commutative : Zariski--Samuel
\end{itemize}

\begin{itemize}
\item Cours 1 : sections blabla à blabla
\item Cours 2 : sections blabla à blabla
\end{itemize}

\tableofcontents

\newpage

\section{Courbes affines planes}

\subsection{Définition et premiers exemples}

Voici une première tentative de définir la notion de \enquote{courbe
  définie par une équation polynomiale} :
\begin{définition}
  Une \emph{courbe algébrique plane} définie sur un corps $K$ est
  l'ensemble des points $(x,y) \in K^2$ tels que $f(x,y) = 0$, où $f$
  est un polynôme non constant à deux variables et à coefficients dans
  $K$.
\end{définition}
Note : le plan $K^2$ sera souvent noté $\aff^2$, c'est-à-dire l'espace
affine de dimension 2 (sous-entendu : sur le corps K)

Soit $C = \{(x,y)\in K^2 \mid f(x,y)=0\}$. On peut définir :
\begin{itemize}
\item le degré de $C$ : c'est le degré du polynôme $f$
\item la notion d'irréductibilité : $C$ est une courbe irréductible si
  $f$ est un polynôme irréductible.
\end{itemize}
Précisions l'interprétation géométrique de l'irréductibilité : si
$f \in K[x,y]$ est non constant, on peut le décomposer\footnote{De
  façon unique, à permutation et à multiplication par un scalaire
  près.} en facteurs irréductibles :
\[ f = f_1^{k_1} \times \ldots \times f_r^{k_r} \]
Dans ce cas, en posant $C_i = {f_i = 0}$, on a :
\[ C = C_1 \cup \ldots \cup C_r \]
autrement dit, la courbe se décompose en courbes plus simples. Une
courbe irréductible, moralement, ne peut pas se décomposer ainsi.

\paragraph{Cependant,} la définition 1 n'est pas toujours satisfaisante !
Ainsi, considérons $x^2 + y^2 = 0$ sur $\real^2$ ($K = \real$).
\begin{itemize}
\item La courbe est réduite au point $(0,0)$, ce qui intuitivement,
  est un objet de dimension 0 et non 1 !
\item Quel est le degré de cette courbe ? Le même ensemble étant
  défini par $x^4 + y^4 = 0$, il n'y a pas unicité du polynôme dont
  l'ensemble est lieu d'annulation ; le degré est donc mal défini (de
  même pour l'irréductibilité).
\end{itemize}

Pour résoudre ces difficultés, on va vouloir prendre $K$les notions
\emph{algébriquement clos}. Afin d'en constater les avantages,
prouvons d'abord le lemme suivant :
\begin{lemme}
  Soit $K$ un corps\footnote{À ce stade, on ne demande pas au corps
    d'être algébriquement clos. Il peut même être fini, auquel cas la
    conclusion est trivialement vraie.}, $f \in K[x,y]$ irréductible,
  $g \in K[x,y]$ quelconque.

  Alors, si $f \nmid g$, alors le système de deux équations
  $f(x,y) = 0, g(x,y) = 0$ n'a qu'un nombre fini de solutions.
\end{lemme}
\begin{proof}
  \emph{Astuce} : considérer $f$ et $g$ comme polynômes à une variable
  à coefficients dans\footnote{Rappel : c'est le corps des fractions
    rationnelles en une indéterminée $y$.} $K(y)$, i.e.
  $f, g \in K(y)[x]$. On remarque que
  \begin{enumerate}[(a)]
  \item $f$ est irréductible dans $K(y)[x]$
  \item $f \nmid g$ dans $K(y)[x]$
  \end{enumerate}
  Esquisse de preuve du $(i)$ (le $(ii)$ étant semblable) : supposons
  que $f = pq$ dans $K[x,y]$. En multipliant par le dénominateur
  commun des coefficients de $p$ dans $K(y)$, puis par celui de
  $q$, on obtient
  \[ a(y)f(x,y) = p'(x,y)q'(x,y) \]
  où $p', q' \in K[x,y]$ et $a \in K[y]$. Ce qui signifie que
  $f \mid p'q'$ dans $K[x,y]$. Or, par hypothèse, $f$ est irréductible
  dans $K[x,y]$, qui est un anneau factoriel. Donc, parmi $p$ et $q$,
  l'un des deux est divisible par $f$ et l'autre divise donc $a(y)$,
  c'est-à-dire qu'il est dans $K[y]$ : dans $K(y)[x]$, il a le statut
  de constante. On a montré que toute factorisation de $f$ dans
  $K(y)[x]$ était triviale.

  Une fois $(i)$ et $(ii)$ établis, on sait que $f$ et $g$ sont
  premiers entre eux dans $K(y)[x]$, donc il existe $u, v \in K(y)[x]$
  tels que $uf+vg=1$. Encore une fois, on multiplie par le
  dénominateur commun et on aboutit à une égalité de la forme
  \[ u'(x,y)f(x,y) + v'(x,y)g(x,y) = h(y) \]
  où $u', v' \in K[x,y]$ et $h \in K[y]$.

  Si $(x_0, y_0)$ est solution du système, alors $y_0$ est racine du
  polynôme à une variable $h$. Il existe donc un nombre fini de $y_0$
  possibles, et pour chacun d'entre eux, ???
\end{proof}

On se souvient maintenant que les corps algébriquement clos sont tous
\emph{infinis} (exercice), ce qui permet d'obtenir :
\begin{corollaire}
  Sur un corps $K$ algébriquement clos, tout polynôme
  \emph{irréductible} dans $K[x,y]$ est déterminé par sa courbe dans
  $\aff^2$ de façon unique, à un facteur constant près.
\end{corollaire}
\begin{proof}
  Toute courbe $\{f(x,y)=0\}$ dans $\aff^2$ sur un corps
  algébriquement clos a un nombre infini de points : si on fixe $x_0$
  de sorte que $f(x_0,y)$ ne soit pas constant non nul en $y$, il
  existe forcément une racine $y_0$, et il y a une infinité de $x_0$
  possibles ? (on utilise donc deux fois le fait que $K$ soit
  algébriquement clos).

  Ainsi, soit $f \in K[x,y]$ irréductible, et $g \in K[x,y]$
  définissant la même courbe. Ils ont alors une infinité de zéros
  communs, donc d'après le lemme précédent, $f \mid g$.
\end{proof}

Ceci permet un choix canonique de polynôme pour chaque courbe
irréductible. On généralise aux courbes quelconques par un résultat
analogue sur les polynômes \emph{sans facteur multiple}. On a donc, en
restreignant les corps considérés, résolu les soucis avec la
définition (une alternative aurait été de définir algébriquement la
notion de courbe de façon plus sophistiquée).

\paragraph{Un autre avantage des corps algébriquement clos} : le nombre de
racines\footnote{Comptées avec multiplicité, bien sûr.} d'un polynôme
est \emph{exactement} égal au degré du polynôme. Plus généralement,
sans clôture algébrique, on perd beaucoup d'information sur le nombre
d'intersections entre courbes. En témoigne cet énoncé approximatif du
\emph{théorème de Bezout}, qu'on reverra plus tard :

\textit{Soient $C_1$ et $C_2$ deux courbes dans $\aff^2$, de degrés
  respectifs $d_1$ et $d_2$. Si $C_1 \cap C_2$ est fini, alors son
  cardinal est majoré par $d_1 d_2$, et c'est même une égalité sous
  certaines conditions de généricité.}

Dans $\real^2$, ceci est \emph{faux}, on peut penser par exemple aux
nombreux cas pour l'intersection de deux coniques. Attention,
\emph{même la majoration est fausse} ! Elle devient vraie si on
remplace l'hypothèse donnée, c'est-à-dire \enquote{le nombre de points
  d'intersection sur $\real$ est fini}, par \enquote{il n'existe pas
  de \emph{composante commune}} (ceci sera vu plus loin), le premier
n'impliquant pas le second. (\textit{Exercice} : trouver un contre-exemple.)

\paragraph{}À partir de maintenant, \textbf{sauf mention du contraire, le corps de
  base sera pris algébriquement clos}. Ce cadre permet quand même de
faire de la géométrie sur $\real$, en travaillant dans $\complex$ tout
en préservant l'invariance par la conjugaison complexe.

\paragraph{Un exemple \enquote{scolaire}} : la droite polaire d'un point et d'un
cercle.

(Faire un dessin.) La construction géométrique par les
tangentes échoue lorsque $P$ est à l'intérieur du
cercle. Algébriquement, on trouve deux points de tangence dans
$\complex\setminus\real$, qui sont \emph{conjugués}. La droite
complexe qui les relie dans $\complex^2$ intersecte $\real^2$ et on
retrouve une droite polaire dans le plan usuel, \emph{algébriquement}.

\paragraph{}Un autre cas intéressant qui dépasse le cadre algébriquement clos (et
ne sera donc pas traité dans ce cours) est $K = \mathbb{Q}$. Exemple
célèbre : la courbe de Fermat $x^n + y^n = 1$, qui n'admet pas de
points rationnels pour $n \geq 3$.

\subsection{Courbes rationnelles}

\emph{Attention}, la terminologie ne signifie \emph{pas} que ces
courbes sont définies sur $\mathbb{Q}$ !

\paragraph{Exemple} Considérons la courbe définie par
$y^2 = x^2 + x^3$. Il s'agit d'une cubique irréductible, qui possède
en plus la propriété intéressante suivante : on peut en donner un
paramétrage par deux fonctions rationnelles d'une variable.

Géométriquement, ce qu'on va faire, c'est paramétrer la courbe par la
pente de la droite reliant le point considéré à l'origine ; on peut
voir ça aussi comme une projection sur la droite $x=1$, de centre
l'origine.

FAIRE UN DESSIN

Concrètement, on va poser $y = tx$. On a alors $t^2 x^2 = x^2 +
x^3$. Il y a deux cas à distinguer :
\begin{itemize}
\item $x \neq 0$ : on simplifie, d'où $t^2 = 1 + x$, puis $x = t^2 -
1$ et $y = t(t^2 - 1)$.
\item $x=0$ : OK, $(0,0)$ est un point de la courbe, et on le retrouve
avec le paramétrage précédent lorsque $t=0$.
\end{itemize}
Finalement on obtient bien quelque chose du type
$(x,y) = (\phi(t), \psi(t))$ où $\phi, \psi \in K(t)$.

\begin{définition}
  Soit $C$ une courbe irréductible dans $\aff^2$, définie par un
  polynôme irréductible $f$. On dit que $C$ est \emph{rationnelle}
  s'il existe deux fonctions rationnelles $\phi$ et $\psi$, dont au
  moins une non constante, telles que $f(\phi(t),\psi(t)) = 0$ pour
  tout $t \in K$ où $\phi$ et $\psi$ sont toutes deux définies.
  
  $(\phi, \psi)$ est appelée une \emph{paramétrisation}\footnote{C'est
    comme ça que c'est dit dans le cours, mais \enquote{paramétrage}
    serait peut-être préférable ?} de $C$.
\end{définition}

\paragraph{Exemples} (1) Toute droite dans $\aff^2$ est rationnelle (exercice).\\
(2) Toute conique irréductible dans $\aff^2$ est rationnelle.

Démontrons le (2). (FAIRE UN DESSIN avec une ellipse qui sera
suggestif mais aura peu de sens car on ne travaille en aucun cas sur
$\real$ !)

Soit $P_0 = (x_0, y_0) \in C$, une droite passant par ce point coupe $C$ en
un autre point (sauf pour une droite particulière, la tangente). On va
paramétrer par la pente de la droite, en posant $y-y_0 =
t(x-x_0)$. Substituer $y$ par $y_0 + t(x-x_0)$ dans $f(x,y)$ donne
\[ A(t)x^2 + B(t)x + C(t) = 0, \quad A,B,C \in K[t] \]
Pour $t$ fixé, la droite de pente $t$ passant par $P_0$ intersecte la
conique en deux points, $P_0$ d'abscisse $x_0$, et le point $P$ qu'on
veut paramétrer d'abscisse $x$ : ce sont les deux racines du trinôme
ci-dessus. Par relation coefficients-racines, on en déduit $x$ comme
une fonction rationnelle de $t$, puis $y$ en découle.

Dans le cas particulier de la conique, on voit que le paramétrage est
très intéressant : connaître \emph{un} point rationnel permet de
connaître \emph{tous} les points rationnels.

\paragraph{Quelles courbes sont rationnelles ?} Reformulation (????)
avec le corps des fonctions rationnelles.

Soit $C$ une courbe irréductible dans $\aff^2$ (lieu d'annulation du
polynôme irréductible $f$), on va définir un corps $K(C)$ contenant
les \enquote{fonctions rationnelles} définies sur $C$. On considère
les fonctions rationnelles
\[ u(x,y) = \frac{p(x,y)}{q(x,y)} \text{ telles que } f \nmid q \]
Notre but étant d'évaluer $u$ en des points de $C$, on souhaite que le
dénominateur ne soit pas identiquement nul sur $C$. Posons
\[ u_1 = \frac{p_1}{q_1} \sim u_2 = \frac{p_2}{q_2} \text{ ssi } f
\mid (p_1q_2 - p_2q_1 \]
On vérifie qu'il s'agit d'une relation d'équivalence, et que le
quotient de notre sous-ensemble de $K(x,y)$ par $\sim$ est bien un
corps. De plus, si $u_1 \sim u_2$, pour tout $p \in C$ où $u_1$ et
$u_2$ sont toutes définies, on aura $u_1(p) = u_2(p)$.
\begin{définition}
  Ce quotient est appelé \emph{corps des fonctions rationnelles sur
    $C$} et noté $K(C)$.
\end{définition}
\emph{Attention}, on peut avoir $u_1 \sim u_2$, $u_1$ définie en $p$,
et $u_2$ \emph{non} définie en $p$ !

\paragraph{Exemple} Sur le cercle unité $x^2 + y^2 = 1$, on peut
vérifier que
\[ u_1(x,y) = \frac{1-y}{x} \sim \frac{x}{1+y} = u_2(x,y) \]
Au point $(0,1)$, $u_1$ n'est pas définie, mais $u_2$ l'est, je peux
donc évaluer la classe de $u_1$ et $u_2$ dans $K(C)$ au point $(0,1)$
en choisissant le représentant $u_2$.

\begin{définition}
  Soit $C$ une courbe irréductible sur un corps $K$. On dit qu'un point
  $p \in C$ est \emph{régulier} pour un élément $\xi \in K(C)$ si,
  dans la classe d'équivalence $\xi$, on peut choisir un représentant
  $g$ qui soit défini en $p$.
\end{définition}

(Fin du cours 1, début du cours 2)

\paragraph{Exercice}
Considérons la courbe $C$ définie par $y = 0$.
Montrer que $K(C)$ peut être identifié à $K(x)$ (corps des fonctions
rationnelles en une indéterminée $x$).

Cette courbe nous fournit aussi un exemple de point non régulier :
$\frac{1}{x}$ n'est pas définie en $0$, et aucun autre représentant de
sa classe n'est défini en $0$.

\begin{propriété}
  Soit $C \subset \aff^2(K)$ une courbe irréductible. Alors $C$ est
  rationnelle si et seulement si $K(C)$ est isomorphe\footnote{Il
    s'agit d'un isomorphisme d'extensions de $K$, c'est-à-dire d'un
    isomorphisme de corps qui fixe le sous-corps commun $K$.} à
  $K(t)$.
\end{propriété}
\textit{Remarque} (pour ceux qui connaissent les extensions de
corps)~: $K(C)$ est de degré de transcendance 1 sur $K$. La
proposition affirme que $C$ est rationnelle ssi l'extension est
transcendante pure.

\begin{proof}
  \textit{Sens $\Rightarrow$.} On considère une paramétrisation
  $\phi(t), \psi(t)$ de notre courbe $C$. On va construire un
  plongement $\iota : K(C) \hookrightarrow K(t)$. Ce qu'on a envie de
  faire, c'est poser
  \[ \iota : K(C) \ni u(x,y) = \frac{p(x,y)}{q(x,y)} \mapsto
  \frac{p(\phi(t),\psi(t))}{q(\phi(t),\psi(t))} \]
  Il faut s'assurer que cette application est bien définie, donc
  \begin{itemize}
  \item qu'elle est compatible avec la relation d'équivalence : c'est
    immédiat en écrivant $f(\phi(t),\psi(t)) = 0$
  \item que $q(\phi(t),\psi(t))$ n'est pas identiquement nulle : il
    suffit pour cela de remarquer qu'il existe un nombre infini de
    points de $C$ de la forme $(\phi(t),\psi(t))$, en effet
    \begin{itemize}
    \item l'intersection des domaines de définition de $\phi$ et
      $\psi$ est infinie (même cofinie, car intersection de deux
      parties cofinies : les dénominateurs de $\phi$ et $\psi$ n'ont
      qu'un nombre fini de racines)
    \item chaque point de $C$ a un nombre fini d'antédédents par
      $(\phi, \psi)$
    \end{itemize}
  \end{itemize}
  Le morphisme défini fixe clairement $K$. En particulier, il n'est
  pas identiquement nul, donc il est injectif (son domaine étant un
  corps). \emph{Attention}, il n'est pas forcément \emph{surjectif} !
  Cependant, il réalise un isomorphisme entre $K(C)$ et un sous-corps
  $\iota(K(C)) \subset K(t)$ ; pour conclure, il suffit d'appliquer le
  \emph{théorème de Lüroth}\footnote{Il ne sera pas démontré en cours,
    on se référera à Zariski--Samuel pour une preuve.} : toute
  sous-extension $K \subsetneq L \subset K(t)$ est isomorphe à $K(t)$.

  \textit{Sens $\Leftarrow$.} On suppose que $K(C) \simeq K(t)$. Les
  (classes d'équivalences des) fonctions rationnelles $x$ et $y$ dans
  $K(C)$ sont envoyées par l'isomorphisme respectivement sur
  $\phi(t), \psi(t) \in K(t)$, au moins l'une des deux n'étant pas
  constante.

  Soit $f$ le polynôme irréductible définissant $C$. Dans $K(C)$, on a
  $f(x,y)=0$, cette égalité étant à lire comme \enquote{le polynôme
    $f$ évalué en les éléments $x, y \in K(C)$ a pour valeur la
    fonction rationnelle nulle}. En transférant par l'isomorphisme on
  a $f(\phi(t),\psi(t))=0$ dans $K(t)$. Ce qui montre que
  $(\phi(t), \psi(t))$ est une paramétrisation de $C$.
\end{proof}

Ce dernier sens de la démonstration donne un paramétrage particulier
de $C$, qui vérifie deux propriétés importantes :
\begin{enumerate}[(1)]
\item Tout point de $C$, sauf un nombre fini, peut être représenté
  comme $(\phi(t),\psi(t))$ pour une certaine valeur de $t \in K$.
\item À l'exception d'un nombre fini de points, une telle
  représentation est unique.
\end{enumerate}
Pour prouver cela, posons $\chi(x,y)$ qui correspond à $t$ par
l'isomorphisme. Alors
\begin{enumerate}[(i)]
\item $(x,y) = (\phi(\chi(x,y)), \psi(\chi(x,y)))$
\item $t = \chi(\phi(t),\psi(t))$
\end{enumerate}
L'égalité $(i)$ prouve $(1)$ car $\chi$ n'a qu'un nombre fini de
points de non-définition sur $C$, en vertu du lemme 1 (considérer le
système $f(x,y) = \chi(x,y) = 0$). De la même façon, $(2)$ découle de
$(ii)$.

\paragraph{Remarque} Ces propriétés ne sont absolument pas vérifiées
par \emph{toutes} les paramétrisations ! Mais on peut, à partir d'une
paramétrisation quelconque, récupérer une \enquote{bonne}
paramétrisation en effectuant la démonstration dans le sens
$(\Rightarrow)$, en utilisant l'isomorphisme du théorème de Lüroth, et
en appliquant le sens $(\Leftarrow)$. Si on vérifie déjà $(1)$ et
$(2)$, $\iota$ devrait être surjectif et on devrait retomber sur nos
pas ; à vérifier.

\paragraph{Exemple} On reprend l'exemple $y^2 = x^2 + x^3$, qu'on
avait paramétré par $x = t^2 - 1$, $y = t(t^2 - 1)$. Cette
paramétrisation vérifie les propriétés $(1)$ et $(2)$, mais on peut la
rendre plus mauvaise : par exemple, faire $t \leftarrow t^2$ (en
obtenant ainsi $x = t^4 - 1$, etc.) nous fait perdre l'unicité de la
représentation. Rien ne nous interdit, dans la définition d'un
paramétrage, de parcourir la courbe deux fois !

\subsection{Applications rationnelles}

\begin{définition}
  Soient $C$ et $D$ deux courbes irréductibles dans
  $\aff^2(K)$. Soient $u, v \in K(C)$. Si pour tout $p \in C$ régulier
  pour $u$ et $v$, $(u(p), v(p)) \in D$, on dit que $\Phi = (u,v)$ est
  une \emph{application rationnelle} de $C$ dans $D$.
\end{définition}
\paragraph{Remarque/notation} Une application rationnelle n'est pas
forcément une application (totale) au sens ensembliste ! On utilisera
la notation $\Phi : C \dashrightarrow D$ pour garder à l'esprit que
$\Phi$ peut ne pas être définie sur certains points.

\paragraph{Exemples} à traiter en exercice :
\begin{itemize}
\item La projection d'une conique irréductible sur une droite, à
  partir d'un point appartenant à la conique, est une application
  rationnelle.
\item Toute paramétrisation d'une courbe algébrique est une
  application rationnelle $\aff^1 \dashrightarrow C$.
\end{itemize}

\begin{définition}
  Soient $C$ et $D$ deux courbes irréductibles dans $\aff^2$. On dit
  qu'une application rationnelle $\Phi : C \dashrightarrow D$ est un
  \emph{isomorphisme birationnel} s'il existe une application
  rationnelle $\Psi : D \dashrightarrow C$ telle que $\Phi \circ \Psi$
  et $\Psi \circ \Phi$ coïncident avec $id_D$ et $id_C$ respectivement
  partout où elles sont définies, avec $\Psi$ non constante.
\end{définition}
\paragraph{Remarque} Cette dernière condition \enquote{$\Psi$ non
  constante} est là pour exclure des cas dégénérés où on ne souhaite
pas dire que $C$ et $D$ sont birationnellement isomorphes. Avec cette
définition, on montre alors que si $\Phi$ est un isomorphisme
birationnel, $\Phi$ est non constant (c'est bien ça ?) Autre
remarque~: si $\Psi : D \dashrightarrow C$ est constante, elle est
définie partout, car on peut choisir un représentant de la classe
d'équivalence qui est défini partout. (Tout ceci est du pinaillage pas
très intéressant.)

\begin{propriété}
  Si $\Phi : C \dashrightarrow D$ est un isomorphisme birationnel,
  alors pour tout point $q \in D$, $\Phi^{-1}(q)$ est fini.
\end{propriété}
\begin{proof}
  Exercice.
\end{proof}
\begin{corollaire}
  Si $\Phi : C \dashrightarrow D$ est un isomorphisme birationnel,
  alors $\Phi$ établit une bijection entre
  \[ U \cap \Phi^{-1}(V) \text{ et } V \cap \Psi^{-1}(U) \]
  et leurs complémentaires respectifs dans $C$ et $D$ sont finis, où
  \begin{itemize}
  \item $\Psi$ est la réciproque
  \item $U \subset C$ est le domaine de définition de $\Phi$
  \item $V \subset D$ est le domaine de définition de $\Psi$
  \end{itemize}
\end{corollaire}

(Fin du cours 2.)

\end{document}

